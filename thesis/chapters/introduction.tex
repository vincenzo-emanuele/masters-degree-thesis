\phantomsection
%\addcontentsline{toc}{chapter}{Introduzione}
\chapter{Introduction}
\markboth{Introduction}{}
% [titolo ridotto se non ci dovesse stare] {titolo completo}

\begin{citazione}

\end{citazione}
\newpage

\section{Overview and objectives} \setstretch{1.3}
This research work aims to analyze and evaluate the utilization of the Data Center efficiency parameter \emph{PUE (Power Usage Effectiveness)} for real world cyberattacks detection. In particular, two attack scenarios will be studied and simulated, namely \emph{DoS (Denial of Service)} and cooling system attack, in order to establish if there is a significant variation in the \emph{PUE} that leads to their detection. 


\subsection{PUE definition}
\emph{PUE (Power Usage Effectiveness)} is a Data Center efficiency parameter introduced by a non-profit consortium called \emph{The Green Grid} in 2007 \cite{avelar2012pue}. It is defined as the ratio of total facility energy to IT equipment energy (equation \ref{eq:pue}):
\begin{displaymath} \label{eq:pue}
    PUE = \frac{Total Facility Energy}{IT Equipment Energy}
\end{displaymath}
\emph{Total Facility Energy} is defined as the energy consumed by the whole Data Center (including IT equipment, power delivery components, cooling system components and data center lighting), while \emph{IT Equipment Energy} is defined as the energy consumed to run the facility's IT infrastructure. \emph{PUE} value can range from 1 to infinity. A \emph{PUE} value of 1 represents an ideal scenario where all the power consumed is used by the IT equipment, making it highly efficient. As the \emph{PUE} value increases above 1, it indicates that a greater portion of the total power is consumed by non-IT equipment, which reduces the overall energy efficiency of the Data Center. Since this parameter provides an insight of the Data Center efficiency, describing how much energy is used by the IT equipment, it has been recognized globally as the industry's most used infrastructure efficiency metric for Data Centers. However, \emph{PUE} depends on many attributes such as Data Center design and implementation, making it difficult to compare Data Centers based on public \emph{PUE} reports. For this purpose, \emph{The Green Grid} has provided a set of guidelines for organizations about making public claims regarding Data Centers \emph{PUE} in order to make this parameter suitable for comparisons. Overall, \emph{PUE} provides an insight about Data Centers energy efficiency helping designers to improve operational efficiency, potentially through comparisons with similar Data Centers. 

\subsection{PUE calculation} \label{subsection:puecalculation}
\emph{Avelar et al.} \cite{avelar2012pue} provides a three-level approach for measuring \emph{PUE}. Each level provides a certain level of detail in \emph{PUE} misuration, considering additional measurement points in order to provide further insight into Data Center infrastructure components' energy consumption. There are three main parameters that vary depending on the chosen level of measurement, namely: \emph{IT Equipment Energy}, \emph{Total Facility Energy} and \emph{Measurement Interval}. 

\subsubsection{Level 1: Basic}
With a level 1 measurement, the parameters are calculated as follows:
\begin{itemize}
    \item \textbf{IT Equipment Energy}: it is measured at the output of the \emph{UPS (Uninterruptible Power Supply)} equipment;
    \item \textbf{Total Facility Energy}: it is measured from the utility service entrance that supplies power to all the equipment within the Data Center;
    \item \textbf{Measurement Interval}: power measurements are performed once a month.
\end{itemize}

\subsubsection{Level 2: Intermediate}
With a level 2 measurement, the parameters are calculated as follows:
\begin{itemize}
    \item \textbf{IT Equipment Energy}: it is measured at the output of the \emph{PDUs (Power Distribution Unit)} equipment;
    \item \textbf{Total Facility Energy}: it is measured from the utility service entrance that supplies power to all the equipment within the Data Center;    
    \item \textbf{Measurement Interval}: power measurements are performed once a day.
\end{itemize}

\subsubsection{Level 3: Advanced}
With a level 3 measurement, the parameters are calculated as follows:
\begin{itemize}
    \item \textbf{IT Equipment Energy}: it is measured at each component of the Data Center excluding non-IT loads;
    \item \textbf{Total Facility Energy}: it is measured from the utility service entrance that supplies power to all the equipment within the Data Center;    
    \item \textbf{Measurement Interval}: power measurements are performed at least every 15 minutes.
\end{itemize}

\subsection{Denial of Service}
\emph{DoS (Denial of Service)} attack is a cyberattack that aims to make a resource unavailable to its users. Usually this kind of attack is performed either by flooding the hosts that provide a specific service until they are unable to respond to their legitimate users or by exploiting a vulnerability of the target. There are various \emph{DoS} attack techniques that have been identified and categorized during the last years, such as \emph{SYN Flood} and \emph{Smurf} \cite{understandingdos}. Nowadays, the most commonly used technique is the \emph{Distributed Denial of Service (DDoS)} attack, which is a variation of \emph{DoS} where the attack is carried out by multiple machines that are under the control of the attacker who is usually able to take control of several machines by exploiting security weaknesses \cite{understandingdos}. Overall, \emph{DoS} lead to damage in terms of time, money and reputation for an organization, so implementing network monitoring mechanism aimed at detection of \emph{DoS} attacks is crucial. 
\subsection{Cooling system attack}
Cooling system attacks pose severe security risks to Data Centers. Over the years, several authors have focused their attention on this specific kind of attacks, analyzing various aspects such as the threat model and the impact of thermal attacks. \emph{Zhihui Shao et al.} \cite{hiddenthreatthermalattacks} described the attacker's capabilities in a thermal attack scenario, indicating that the attacker runs power-intensive applications to increase server power consumption, resulting in a single server equipped with multiple CPUs and GPUs consuming a large amount of energy. These authors also divided the impact of thermal attacks in two main cathegories, namely:
\begin{itemize}
    \item \textbf{Performance degradation}: when the server temperature exceeds a certain threshold a thermal emergency occurs. In this scenario the server runs in a low power state mode to avoid the temperature from rising further, which could lead to hardware damage. It is evident that this remediation strategy results in performance degradation; 
    \item \textbf{System outage}: when a thermal emergency occurs and the server temperature keeps rising despite the load capping, the system automatically shuts down to prevent hardware damage. The consequences of a system outage depend on the nature of the application and may become catastrophic in the case of latency-critical applications.
\end{itemize}
The aforementioned work also explores thermal attack strategies and their feasibility aiming to describe a real-world attack scenario and highlight the motivations behind a thermal attack. 

\section{Motivations}
There are two main reasons that led to the study of techniques based on \emph{PUE} monitoring for cyberattacks detection. The first motivation concerns the ease of \emph{PUE} calculation that does not require specific equipment apart from wattmeters that should be placed at various points within the Data Center, depending on the level of granularity desired in \emph{PUE} calculation (as described in section \ref{subsection:puecalculation}). The second motivation is closely related to the cooling system attack scenario as it highlights the reasons why it is essential to be concerned about potential attacks on the Data Centers' management system. In recent years, several research studies have explored various methodologies to integrate Cloud and IoT solutions in the context of Data Centers monitoring and management. In 2016 \emph{Q Liu et al.} \cite{liu2016green} proposed an air conditioning system that includes both IoT sensors and cloud-based systems for Data Center management. Moreover, in 2020 \emph{Ramphela et al.} \cite{ramphela2020internet} proposed an integrated monitoring system for Data Centers based on the development of various subsystems, such as an embedded system that includes sensors that gather monitoring data. In this context, the pervasive usage of IoT and Cloud technologies exposes Data Centers to additional risks. As reported in the research work by \emph{Francesca Meneghello et al.} \cite{meneghello2019iot}, the main risk associated with the IoT devices comes from the lack of the implementation of security mechanisms especially in cheaper devices that are widely spread. An unauthorized access to the Data Center's control system could lead to the modification of the cooling system's operational parameters. In cases where such violation may go unnoticed by the Intrusion Detection or Intrusion Prevention System, the proposed approach based on monitoring energy parameters can be a viable option for detecting such unauthorized access.


\section{Results}
The achieved results suggest that a correlation between \emph{PUE} fluctuations and the presence of cyberattacks exists in both simulated scenarios. In particular, in the \emph{DoS} scenario, there is a substantial variation in \emph{PUE} up to a certain load threshold (approximately 80\%), while in the cooling system attack scenario, the \emph{PUE} variation is independent of the Data Center load, making the proposed approach particularly suitable for the detection of this specific attack. A detailed analysis of the results achieved is presented in Chapter \ref{chapter:research_study}, where the entire research work process is comprehensively discussed.

\section{Structure of the Thesis}
The Thesis is structured as follows:
\begin{itemize}
    \item \textbf{Chapter 2 (State of the Art):} this chapter provides an overview about the currently available techniques of \emph{DoS} and cooling system attack detection. Furthermore, in order to choose a viable platform to perform Data Center simulations, several Cloud simulators will be analyzed and compared;
    \item \textbf{Chapter 3 (GreenCloud Simulator Overview):} this chapter explores in depth the \emph{GreenCloud} simulator as it has been chosen as the reference tool for the simulations;
    \item \textbf{Chapter 4 (Data Center Design):} this chapter describes the architecture, the energy model, the IT capacity and the power and cooling parameters of the virtual Data Center used for the performed simulations;
    \item \textbf{Chapter 5 (Research Study):} this chapter illustrates the research study process, starting from the changes implemented in \emph{GreenCloud} in order to make it suitable for this work and then describing the performed simulations and the \emph{PUE} calculation. The chapter concludes with a discussion on the achieved results, providing an idea of how the variation of PUE can be a suitable parameter for attacks detection;
    \item \textbf{Chapter 6 (Conclusions and Future Work):} this chapter summarizes the research study and discusses potential future works.
\end{itemize}