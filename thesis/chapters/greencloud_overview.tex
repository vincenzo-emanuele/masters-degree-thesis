\phantomsection
%\addcontentsline{toc}{chapter}{Introduzione}
\chapter{GreenCloud Simulator Overview}
\markboth{GreenCloud Simulator Overview}{}
% [titolo ridotto se non ci dovesse stare] {titolo completo}

\begin{citazione}

\end{citazione}
\newpage

\section{Selection of GreenCloud} 
Based on the comparisons made among the different analyzed simulators, each of them has its strengths and weaknesses. However, for the needs required in the study addressed by this thesis work, it is essential to prioritize aspects related to the energy consumption of the computing center and the accuracy of simulations. From the conducted comparisons, it is evident that \emph{GreenCloud} is the simulator that accurately considers these aspects as it is built on top of \emph{NS2} simulator and fully implements the \emph{TCP/IP} protocol. Moreover, \emph{GreenCloud} offers its users a set of features related to the simulation and to the energy consumption management. In particular, it is possible to choose between several pre-implemented Data Center architectures and energy models as well as to customize them. Furthermore \emph{GreenCloud} provides various workload scheduling and power saving models, allowing programmers to implement new ones. Despite \emph{GreenCloud} simulation times tend to be high, for the purposes of this study it is reasonable to prioritize granularity over performance. Therefore, the study will continue using \emph{GreenCloud} as the reference tool for the simulations to be conducted.


\section{Available Data Center architectures}
As mentioned in the previous section, \emph{GreenCloud} provides several Data Centers architectures. The implemented architectures consist of various components, described as follows:
\begin{itemize}
    \item \textbf{Servers: } single core nodes with a fixed processing power limit expressed in \emph{MIPS} (million instructions per second) or \emph{FLOPS} (floating point operations per second) that are responsible for task execution. These components are organized in racks and the architecture includes the presence of a Top-of-Rack switch that connects them to the access layer of the architecture;
    \item \textbf{Workloads: }
\end{itemize}

\section{Energy model of switches, CPU, memory and disks}

\section{Available power saving models}

\section{Workload scheduling algorithms comparison} 
    
