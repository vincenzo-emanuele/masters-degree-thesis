\phantomsection
%\addcontentsline{toc}{chapter}{Introduzione}
\chapter{Conclusions and Future Work}
\markboth{Conclusions and Future Work}{}
% [titolo ridotto se non ci dovesse stare] {titolo completo}

\begin{citazione}
This chapter summarizes the research study
and discusses potential future works.
\end{citazione}
\newpage

\section{Conclusions} 
The idea behind this research work originated from a reflection about the usage of the \emph{PUE} parameter which is mainly used by companies as an indicator of their data centers' efficiency. The aim was to investigate the relationship between the variation of this parameter and the presence of a cyberattack, providing a practical importance to a parameter that is primarily used in a commercial and promotional way. After studying the \emph{PUE} metric, through the exploration of its introduction, usage and calculation, two attack scenarios to be evaluated, namely the \emph{DoS} and the cooling system attack, were chosen. Subsequently, an investigation about the techniques employed in the state of the art to detect the chosen attacks was carried out in order to contextualize the proposed approach into the modern techniques. Since this research required data about the energy consumption of Data Centers under various load and cooling conditions, a close attention was payed to the selection of a Data Center simulator that was able to provide accurate energy reports. The \emph{GreenCloud} simulator has been chosen as the reference platform for the simulations that were carried out in order to gather the data that revealed a correlation between the \emph{PUE} variation and the presence of the above-mentioned attacks. The results can be summarized as follows:
\begin{itemize}
    \item \emph{DoS} attack detection through \emph{PUE} variation monitoring requires a workload profile in order to distinguish between an expected load increase and an attack. Depending on the workload level the \emph{PUE} variation can be more or less observable. When the load goes from 40\% to 80\% a \emph{PUE} variation of 23\% is observable, while when the load goes from 80\% to 100\% only a \emph{PUE} variation of 6.66\% can be observed. Overall, the \emph{DoS} detection through \emph{PUE} variation monitoring is possible since the \emph{PUE} variation is not negligible; nevertheless such a detection algorithm, whose implementation goes beyond the scope of this research work, should be implemented in order to validate this methodology;
    \item Cooling system attacks detection through \emph{PUE} variation monitoring is definitely more feasible than the \emph{DoS} attack detection as this problem is reduced to determining which \emph{PUE} curve the points observed during runtime belong to, so it only requires the calculation of \emph{PUE} curves under normal and altered conditions. Overall, there is a strong correlation between \emph{PUE} variation and cooling system attack and this approach can also be employed to detect faults in the cooling system since they occur under the same conditions as the attacks. Therefore, in the case of detection, it is important to distinguish between an attack and a fault, even though both situations require urgent intervention. 
\end{itemize}
It should be clear that the goal of this research work is not to provide the implementation of a detection algorithm based on \emph{PUE} variation monitoring. However, an eventual detection algorithm based on \emph{PUE} monitoring should be integrated into a more complex security ecosystem in order to bring better results. This leads to the consideration that the aim of this work is to propose an approach that could bring to the implementation of a system that integrates with the current technologies. The choice of the parameter to monitor for attack detection fell on \emph{PUE} as it is easy to calculate, due to the cost-effectiveness of the tools used for its computation. Moreover, this provides the basis to design and implement an architecture specialized in monitoring the energy consumption of the Data Center that, regardless of the specific issue addressed in this research, results effective to spread an `energy culture' which is useful for reducing operational costs and emissions in Data Centers. 
\section{Future Work}
As highlighted on multiple occasions, this research work was conducted under certain assumptions and simplifications, some of which were useful to focus only on relevant issues, while others could be overcome to obtain data more in line with real scenarios. Three future work ideas are described below:
\begin{itemize}
\item Servers should not be modeled as single-core machines but as multi-core machines;
\item The linear model for energy consumption is simplified, and more complex models in line with those used by modern processors should be considered;
\item \emph{GreenCloud} lacks a module for managing the thermal aspects of the data center and for PUE calculation. Therefore, it would be useful to implement a module for handling these aspects. However, this is not straightforward, as it would require knowledge of the conditioning models, which is a complex field.
\end{itemize}
Thanks to the open-source nature of \emph{GreenCloud}, it is possible to make these modifications and additions to the simulator. Finally, as mentioned before, this research work opens the door for the implementation of attack detection algorithms based on \emph{PUE} variation monitoring that could be used to improve Data Centers security. The implementation of these algorithms could provide insights about the potential and limitations of the proposed approach through a real-world testing within actual security ecosystems, helping to understand the practical benefits of this methodology.